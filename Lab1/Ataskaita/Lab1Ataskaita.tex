\documentclass[a4paper,10pt]{article}
\usepackage[utf8]{inputenc}
\usepackage{amsmath}
\usepackage{graphicx}
\usepackage[margin=1in]{geometry}
\usepackage[T1]{fontenc}
\usepackage{tikz,pgfplots}
\usepackage{indentfirst}
\usepackage{titling}
\usepackage{siunitx}
\usepackage[hidelinks]{hyperref}
\usetikzlibrary{matrix,positioning}
\graphicspath{ {/home/pijus/Documents/Latex/irodymai} }
\renewcommand{\thesection}{\arabic{section}.}
\renewcommand{\thesubsection}{\thesection\arabic{subsection}.}
\renewcommand{\thesubsubsection}{\thesubsection\arabic{subsubsection}.}
\renewcommand\maketitlehooka{\null\mbox{}\vfill}
\renewcommand\maketitlehookd{\vfill\null}
\pgfplotsset{compat=1.12}
\tikzset{My Style/.style={/pgf/number format/fixed,
                  /pgf/number format/precision=3}}
\sisetup{round-mode=places,round-precision=3}
\hypersetup{
    citecolor=black,
    filecolor=black,
    linkcolor=black,
    urlcolor=black
}
%opening
\title{Lygiagretaus programavimo laboratorinių darbų analizė}
\author{Pijus Petkevičius}

\begin{document}
\maketitle
\newpage
\tableofcontents
\newpage

\section{1 laboratorinis darbas}
\subsection{Aprašymas}
\subsubsection{Įžanga}
Aibę $A$ sudaro geografiniai taškai, nurodant platumos ir ilgumos koordinates.
Iš šios aibės reikia parinkti taškų aibę $X$ tokią, kad atstumų nuo kiekvieno aibės $A$ taško iki jam artimiausio aibės $X$ taško suma būtų minimali $X  \subset A $.

Faile lab\_data.dat pateikti 50000 geografinių taškų, kur viena eilutė aprašo vieno geografinio taško koordinates.

Faile lab\_01\_2\_algorithm.cpp pateikta programa, kuri randa nurodyto $n$ taškų aibę $X$, atitinkančią uždavinio sąlygą, naudojant paprastosios atsitiktinės paieškos (angl. Pure Random Search, PRS) algoritmą.

\noindent Pagrindiniai algoritmo parametrai (globalūs kintamieji):
\begin{itemize}
	\item num\_points: duomenų aibės $A$ dydis (max 50000)
	\item num\_variables: ieškomos taškų aibės $X$ dydis
	\item num\_iterations: sprendinio paieškai skirtų iteracijų skaičius (kuo daugiau, tuo didesnė tikimybė rasti geresnį sprendinį).
\end{itemize}

Algoritmų vykdymo pradžioje sudaroma atstumų matrica, kurioje saugomi atstumai kilometrais tarp taškų, suskaičiuoti pagal Haversino formulę.
Atsižvelgiant į tai, kad atstumas nuo taško $a$ iki taško $b$ yra lygus atstumui nuo taško $b$ iki taško $a$, yra užpildoma tik pusė matricos.
Šioje matricoje saugomi atstumai yra naudojami vykdant aibės $X$ taškų paiešką.
\subsubsection{Užduotis}
\begin{enumerate}
	\item Pasirinkti duomenų aibės dydį ir algoritmo iteracijų skaičių, kad atstumų matricos skaičiavimas užtruktų ne mažiau 10 sekundžių, o sprendinio paieškos laikas būtų nemažesnis nei 20 sekundžių. \label{pirmas nurodymas}

	\item Duomenų įkėlimą ir atstumų matricos skaičiavimą laikyti nuosekliąja algoritmo dalimi, o sprendinio paiešką - lygiagretinama dalimi, įvertinti teorinius galimus algoritmo pagreitėjimus naudojant 2 ir 4 procesorius, bei didžiausią galimą pagreitėjimą. \label{antras nurodymas}

	\item Duomenų įkėlimą ir atstumų matricos skaičiavimą laikyti nuosekliąja algoritmo dalimi, sudarykite lygiagretųjį bendros atminties algoritmą ir eksperimentiniu būdu ištirkite jo pagreitėjimą naudodami 2 ir 4 procesorius. \label{trecias nurodymas}

	\item Sudarykite lygiagretų bendros atminties algoritmą atstumų matricos skaičiavimui ir eksperimentiniu būdu ištirkite jo pagreitėjimą naudodami 2 ir 4 procesorius. \label{ketvirtas nurodymas}

	\item Pananalizuoti, kai matricos reikšmių suskaičiavimą lygiagrečiąja dalimi, o pure random search (PRS), nuosekliąja. \label{penktas nurodymas}

\end{enumerate}
\newpage

\subsection{Kompiuterinės įrangos ir parametrų pasirinkimas}
Algoritmo analizei buvo naudojama \textbf{Apple Mac Mini Desktop Computer, 3.2GHz 6-Core Intel Core i7} kompiuteris, kurio dėka, buvo galima paleisti ant 2, 4 ir 6 procesorių.
Kad įgyvendinti \textbf{\ref{pirmas nurodymas}} nurodymą, buvo pasirinkta:
\begin{itemize}
	\item num\_points = 12000
	\item num\_iterations = 30000
\end{itemize}

\begin{center}
	\begin{tabular}{| c | c | c |}
		\hline
		Duomenų nuskaitymas (s) & Atstumų matricos skaičiavimas (s) & PRS skaičiavimas (s) \\
		\hline
		\num{0.00323701}        & \num{10.3124}                     & \num{19.9546}        \\
		\hline
		\num{0.00437999 }       & \num{10.3154}                     & \num{19.993}         \\
		\hline
		\num{0.00339818 }       & \num{10.3207}                     & \num{19.9673}        \\
		\hline
	\end{tabular}
\end{center}
\subsection{\ref{antras nurodymas} nurodymo teoriniai įverčiai}
Paleidus programą 3 kartus, gauti skaičiavimo dalių rezultatai:
\begin{center}
	\begin{tabular}{| c | c | c |}
		\hline
		Duomenų nuskaitymas (s)   & Atstumų matricos skaičiavimas (s) & PRS skaičiavimas (s)   \\
		\hline
		\num{0.00367172666666667} & \num{10.3161666666667}            & \num{19.9716333333333} \\
		\hline
	\end{tabular}
\end{center}

Pagal \ref{antras nurodymas} nurodymą, nuosekliąja dalimi $(\alpha)$ laikoma duomenų nuskaitymas ir atstumų matricos skaičiavimas, o lygiagrečiąja $(\beta)$- PRS skaičiavimas.

$$ \alpha = \frac {\text{nuoseklioji dalis}} {\text{visas laikas}} $$
$$ \beta = \frac {\text{lygiagrečioji dalis}} {\text{visas laikas}} $$

Gauname kad: $$\alpha = \num{0.340684615341697}$$ $$ \beta = \num{0.659315384658303} $$

Teorinis pagreitėjimas naudojant $p$ procesorių:
$$S_p = \frac {1} {\alpha + \frac{\beta} {p}} $$
$$S_2 = \frac {1} { \num{0.340684615341697}+ \frac{\num{0.659315384658303}} {2}}  = \num{1.49177515510631}$$
$$S_4 = \frac {1} { \num{0.340684615341697} + \frac{\num{0.659315384658303}} {4}}  = \num{1.97818668769039}$$

Teorinis maksimumas pagal Andalo Dėsnį:
$$ S_{max} = \lim_{p \rightarrow \infty} \frac {1} {\alpha + \frac {\beta} {p}} = \frac {1} {\alpha} = \frac {1} { \num{0.340684615341697}}= \num{2.9352660935306}$$
\newpage
\subsection{Pure random search (PRS)}

pakeisti i pseudo koda
\begin{verbatim}
    int *best_solution = new int[num_variables]; 
    double f_solution, f_best_solution = 1e10; 
    #pragma omp parallel reduction (min: f_best_solution ) private (f_solution)
    #pragma omp for schedule(dynamic)
    for (int i=0; i<num_iterations; i++) {
        int *solution = new int[num_variables];
        random_solution(solution);
        f_solution = evaluate_solution(solution);
        if (f_solution < f_best_solution) { 
            (mazesnis) uz geriausia zinoma
            f_best_solution = f_solution;
            if(f_best_solution == f_solution){
                #pragma omp critical (DataCollection)
                {
                    for (int j=0; j<num_variables; j++) {
                        best_solution[j] = solution[j];
                    }
                }
            }
        }
    }
\end{verbatim}
PRS lygiagretinimas:\\
\begin{center}
	\begin{tikzpicture}[domain=0:3, scale=1]
		\begin{axis}[
				xlabel={Branduolių skaičius},
				ylabel={Algoritmo efektyvumas},
				legend style={at={(0.5,-0.17)},
						anchor=north,legend cell align=left} ]
			%x=y
			\addplot[black, mark=none] table [x=x, y=y,] {
					x y
					1 1
					2 2
					2.3 2.3
				};
			% Teorinis pagreitejimas
			\addplot[blue, mark=triangle*,
				nodes near coords={\pgfmathprintnumber[fixed zerofill, precision=3]{\pgfplotspointmeta}},
				visualization depends on=\thisrow{label} \as \Label] table [x=x, y=y] {
					x y label
					1 1 1
					2 1.49177515510631  1.49177515510631
					4 1.97818668769039  1.49177515510631
					6 2.21940844246444  2.21940844246444
				};
			% Realus pagreitejimas

			\addplot [red, mark=triangle*,
				nodes near coords={\pgfmathprintnumber[fixed zerofill, precision=3]{\pgfplotspointmeta}},
				visualization depends on=\thisrow{label} \as \Label] table [x=x, y=y] {
					x y label
					1 1 1
					2 1.47144861531196 1.47144861531196
					4 1.91794196594863 1.91794196594863
					6 2.11142684438 2.11142684438
				};
			\legend {x=y,Teorinis pagreitėjimas,Realus pagreitėjimas}
		\end{axis}
	\end{tikzpicture}
\end{center}
\newpage
\subsubsection{Atstumų matricos skaičiavimas}

\begin{verbatim}
    #pragma omp parallel for schedule(dynamic)
    for (int i=0; i<num_points; i++) {
        distance_matrix[i] = new double[i+1];
        for (int j=0; j<=i; j++) {
            distance_matrix[i][j] = Haversine_distance(points[i][0], points[i][1], points[j][0], points[j][1]);
        }
    }
\end{verbatim}
Matricos skaiciavimo laikai:\\
\begin{center}
	\begin{tikzpicture}[domain=0:3, scale=1]
		\begin{axis}[
				xlabel={Branduolių skaičius},
				ylabel={Algoritmo efektyvumas},
				legend style={at={(0.5,-0.17)},
						anchor=north,legend cell align=left} ]
			\addplot[black, mark=none] table [x=x, y=y,] {
					x y
					1 1
					1.4 1.4
				};
			% Teorinis pagreitejimas
			\addplot[blue, mark=triangle*,
				nodes near coords={\pgfmathprintnumber[fixed zerofill, precision=3]{\pgfplotspointmeta}},
				visualization depends on=\thisrow{label} \as \Label] table [x=x, y=y] {
					x y label
					1 1 1
					2 1.20522833024383 1.20522833024383
					4 1.34304363117962  1.34304363117962
					6 1.39626355564776 1.39626355564776
				};
			%dynamix scheduling
			\addplot[red, mark=triangle*,
				nodes near coords={\pgfmathprintnumber[fixed zerofill, precision=3]{\pgfplotspointmeta}},
				visualization depends on=\thisrow{label} \as \Label] table [x=x, y=y] {
					x y label
					1 1 1
					2 1.17528298304024 1.17528298304024
					4 1.31106750410733 1.31106750410733
					6 1.38428949533786 1.38428949533786
				};
			%guided scheduling
			\addplot[orange, mark=triangle*,
				nodes near coords={\pgfmathprintnumber[fixed zerofill, precision=3]{\pgfplotspointmeta}},
				visualization depends on=\thisrow{label} \as \Label] table [x=x, y=y] {
					x y label
					1 1 1
					2 1.14934249820694 1.14934249820694
					4 1.27276503291191 1.27276503291191
					6 1.28140219409186 1.28140219409186
				};
			% static scheduling
			\addplot[violet, mark=triangle*,
				nodes near coords={\pgfmathprintnumber[fixed zerofill, precision=3]{\pgfplotspointmeta}},
				visualization depends on=\thisrow{label} \as \Label] table [x=x, y=y] {
					x y label
					1 1 1
					2 1.06525542297062 1.06525542297062
					4 1.24443394120132 1.24443394120132
					6 1.23684155937314 1.23684155937314
				};
			\legend {x=y,Teorinis pagreitėjimas,Dynamic realus pagreitėjimas,Guided realus pagreitėjimas, Static realus pagreitėjimas}
		\end{axis}
	\end{tikzpicture}
\end{center}
\newpage
\subsection{Rezultatų analizė}
\bigskip

Abu sulygiagrerinti: \\
\begin{center}
	\begin{tikzpicture}[domain=0:3, scale=1]
		\begin{axis}[
				xlabel={Branduolių skaičius},
				ylabel={Algoritmo efektyvumas},
				legend style={at={(0.5,-0.17)},
						anchor=north,legend cell align=left} ]
			%x=y
			\addplot[black, mark=none] table [x=x, y=y,] {
					x y
					1 1
					2 2
					4 4
					6 6
				};
			%teorinis pagreitėjimas
			\addplot[blue, mark=triangle*,
				nodes near coords={\pgfmathprintnumber[fixed zerofill, precision=3]{\pgfplotspointmeta}},
				visualization depends on=\thisrow{label} \as \Label] table [x=x, y=y] {
					x y label
					1 1 1
					2 1.99975760295228 1.99975760295228
					4 3.99854597016622 3.99854597016622
					6 5.99636580611994  5.99636580611994
				};
			% realus pagreitėjimas
			\addplot[red, mark=triangle*,
				nodes near coords={\pgfmathprintnumber[fixed zerofill, precision=3]{\pgfplotspointmeta}},
				visualization depends on=\thisrow{label} \as \Label] table [x=x, y=y] {
					x y label
					1 1 1
					2 1.91173481098984 1.91173481098984
					4 3.70599235210796 3.70599235210796
					6 4.44680617366648 4.44680617366648
				};
			\legend {x=y,Teorinis pagreitėjimas,Realus pagreitėjimas}
		\end{axis}
	\end{tikzpicture}
\end{center}

\end{document}
