\documentclass[a4paper,10pt]{article}
\usepackage[utf8]{inputenc}
\usepackage{amsmath}
\usepackage{graphicx}
\usepackage[margin=1in]{geometry}
\usepackage[T1]{fontenc}
\usepackage{tikz,pgfplots}
\usepackage{indentfirst}
\usepackage{titling}
\usepackage{siunitx}
\usepackage[hidelinks]{hyperref}
\usepackage{newfloat}

\DeclareFloatingEnvironment[fileext=dia]{diagram}

\usetikzlibrary{matrix,positioning}
\graphicspath{ {/home/pijus/Documents/Latex/irodymai} }
\renewcommand{\thesection}{\arabic{section}.}
\renewcommand{\thesubsection}{\thesection\arabic{subsection}.}
\renewcommand{\thesubsubsection}{\thesubsection\arabic{subsubsection}.}
\renewcommand{\figurename}{Pseudokodas}
\renewcommand{\tablename}{Lentelė}
\renewcommand{\diagramname}{Diagrama}
\renewcommand\maketitlehooka{\null\mbox{}\vfill}
\renewcommand\maketitlehookd{\vfill\null}
\pgfplotsset{compat=1.12}
\tikzset{My Style/.style={/pgf/number format/fixed,
                  /pgf/number format/precision=3}}
\sisetup{round-mode=places,round-precision=3}
\hypersetup{
    citecolor=black,
    filecolor=black,
    linkcolor=black,
    urlcolor=black
}
%opening
\title{Lygiagretaus programavimo laboratorinių darbų analizė}
\author{Pijus Petkevičius}

\begin{document}
\maketitle
\newpage
\tableofcontents
\newpage

\section{1 laboratorinis darbas}
\subsection{Aprašymas}
\subsubsection{Įžanga}
Aibę $A$ sudaro geografiniai taškai, nurodant platumos ir ilgumos koordinates.
Iš šios aibės reikia parinkti taškų aibę $X$ tokią, kad atstumų nuo kiekvieno aibės $A$ taško iki jam artimiausio aibės $X$ taško suma būtų minimali $X  \subset A $.

Faile lab\_data.dat pateikti 50000 geografinių taškų, kur viena eilutė aprašo vieno geografinio taško koordinates.

Faile lab\_01\_2\_algorithm.cpp pateikta programa, kuri randa nurodyto $n$ taškų aibę $X$, atitinkančią uždavinio sąlygą, naudojant paprastosios atsitiktinės paieškos (angl. Pure Random Search, PRS) algoritmą.

\noindent Pagrindiniai algoritmo parametrai (globalūs kintamieji):
\begin{itemize}
	\item num\_points: duomenų aibės $A$ dydis (max 50000)
	\item num\_variables: ieškomos taškų aibės $X$ dydis
	\item num\_iterations: sprendinio paieškai skirtų iteracijų skaičius (kuo daugiau, tuo didesnė tikimybė rasti geresnį sprendinį).
\end{itemize}

Algoritmų vykdymo pradžioje sudaroma atstumų matrica, kurioje saugomi atstumai kilometrais tarp taškų, suskaičiuoti pagal Haversino formulę.
Atsižvelgiant į tai, kad atstumas nuo taško $a$ iki taško $b$ yra lygus atstumui nuo taško $b$ iki taško $a$, yra užpildoma tik pusė matricos.
Šioje matricoje saugomi atstumai yra naudojami vykdant aibės $X$ taškų paiešką.
\subsubsection{Užduotis}
\begin{enumerate}
	\item Pasirinkti duomenų aibės dydį ir algoritmo iteracijų skaičių, kad atstumų matricos skaičiavimas užtruktų ne mažiau 10 sekundžių, o sprendinio paieškos laikas būtų nemažesnis nei 20 sekundžių. \label{pirmas nurodymas}

	\item Duomenų įkėlimą ir atstumų matricos skaičiavimą laikyti nuosekliąja algoritmo dalimi, o sprendinio paiešką - lygiagretinama dalimi, įvertinti teorinius galimus algoritmo pagreitėjimus naudojant 2 ir 4 procesorius, bei didžiausią galimą pagreitėjimą. \label{antras nurodymas}

	\item Duomenų įkėlimą ir atstumų matricos skaičiavimą laikyti nuosekliąja algoritmo dalimi, sudarykite lygiagretųjį bendros atminties algoritmą ir eksperimentiniu būdu ištirkite jo pagreitėjimą naudodami 2 ir 4 procesorius. \label{trecias nurodymas}

	\item Sudarykite lygiagretų bendros atminties algoritmą atstumų matricos skaičiavimui ir eksperimentiniu būdu ištirkite jo pagreitėjimą naudodami 2 ir 4 procesorius. \label{ketvirtas nurodymas}

	\item  Ištirti algoritmo pagreitėjimo priklausomybes nuo procesorių skaičiaus, kai matricos skaičiavimas ir sprendinio paieška išlygiagretinti.\label{penktas nurodymas}

\end{enumerate}
\newpage

\subsection{Kompiuterinės įrangos ir parametrų pasirinkimas}
Algoritmo analizei buvo naudojama \textbf{Apple Mac Mini Desktop Computer, 3.2GHz 6-Core Intel Core i7} kompiuteris, kurio dėka, buvo galima paleisti ant 2, 4 ir 6 procesorių.
Kad įgyvendinti \textbf{\ref{pirmas nurodymas}} nurodymą, buvo pasirinkta:
\begin{itemize}
	\item num\_points = 12000
	\item num\_iterations = 30000
\end{itemize}
\begin{table}[ht]
	\begin{center}
		\begin{tabular}{| c | c | c |}
			\hline
			Duomenų nuskaitymas (s) & Atstumų matricos skaičiavimas (s) & PRS skaičiavimas (s) \\
			\hline
			\num{0.00323701}        & \num{10.3124}                     & \num{19.9546}        \\
			\hline
			\num{0.00437999}        & \num{10.3154}                     & \num{19.993}         \\
			\hline
			\num{0.00339818}        & \num{10.3207}                     & \num{19.9673}        \\
			\hline
		\end{tabular}
	\end{center}
	\caption{Algoritmo skaičiavimo dalių rezultatai, naudojant \textbf{Mac Mini} kompiuterį}
\end{table}
\subsection{Teoriniai įverčiai}
Paleidus programą 3 kartus, gauti skaičiavimo dalių rezultatai:
\begin{table}[ht]
	\begin{center}
		\begin{tabular}{| c | c | c |}
			\hline
			Duomenų nuskaitymas (s)   & Atstumų matricos skaičiavimas (s) & PRS skaičiavimas (s)   \\
			\hline
			\num{0.00367172666666667} & \num{10.3161666666667}            & \num{19.9716333333333} \\
			\hline
		\end{tabular}
	\end{center}
	\caption{Nuoseklaus algoritmo skačiavimo dalių rezultatai}
\end{table}

Pagal \textbf{\ref{antras nurodymas}} nurodymą, nuosekliąja dalimi $(\alpha)$ laikoma duomenų nuskaitymas ir atstumų matricos skaičiavimas, o lygiagrečiąja $(\beta)$- PRS skaičiavimas.

$$ \alpha = \frac {\text{nuoseklioji dalis}} {\text{visas laikas}} $$
$$ \beta = \frac {\text{lygiagrečioji dalis}} {\text{visas laikas}} $$

Gauname kad: $$\alpha = \num{0.340684615341697}$$ $$ \beta = \num{0.659315384658303} $$

Teorinis pagreitėjimas naudojant $p$ procesorių:
$$S_p = \frac {1} {\alpha + \frac{\beta} {p}} $$
$$S_2 = \frac {1} { \num{0.340684615341697}+ \frac{\num{0.659315384658303}} {2}}  = \num{1.49177515510631}$$
$$S_4 = \frac {1} { \num{0.340684615341697} + \frac{\num{0.659315384658303}} {4}}  = \num{1.97818668769039}$$

Teorinis maksimumas pagal Andalo Dėsnį:
$$ S_{max} = \lim_{p \rightarrow \infty} \frac {1} {\alpha + \frac {\beta} {p}} = \frac {1} {\alpha} = \frac {1} { \num{0.340684615341697}}= \num{2.9352660935306}$$
\newpage
\subsection{Pure random search (PRS)}
Bandymas išlygiagretinti PRS algoritmą, buvo atliktas 2 būdais.

\textbf{\ref{PRS1}} algoritme buvo pasitelkta \textbf{Dynamic} scheduling strategija, ir reduction min: f\_best\_solution. Kievieną karta, kai randamas geresnė sprendinio reikšmė, ji priskiriama f\_best\_solution kintamajam(jis automatiškai pasiima tik mažesnę reikšmę). Vėliau viskas  geriausio sprendinio reikšmės buvo išsaugojamos naudojant critical žymę:
\begin{figure}[ht]
	\begin{verbatim}
    int *best_solution; 
    double f_solution, f_best_solution; 
    #pragma omp parallel reduction (min: f_best_solution ) private (f_solution)
    #pragma omp for schedule(dynamic)
    for (int i=0; i<num_iterations; i++) {
        // random find and evaluate solution
        f_solution = evaluate_solution(solution);
        if (f_solution < f_best_solution) { 
            f_best_solution = f_solution;
            if(f_best_solution == f_solution){
                #pragma omp critical (DataCollection)
                {
                   // copy solution values to the best_solution array
                }
            }
        }
    }
\end{verbatim}
	\caption{PRS pirmas algoritmas}
	\label {PRS1}
\end{figure}

\textbf{\ref{PRS2}} antrajame algoritme veikimo principas gana panašus, tik ciklas paskirstomas keliems branduoliams, randamas lokali geriausia f\_best\_solution\_tmp reikšmė ir po ciklo ji priskiriama f\_best\_solution ir išsaugojamos geriausio sprendinio reikšmės:
\begin{figure}[ht]
	\begin{verbatim}
    double f_best_solution;
    int *best_solution; 
    #pragma omp parallel reduction(min: f_best_solution) 
    {
        int *best_solution_tmp;
        double f_solution, f_best_solution_tmp;
        #pragma omp for schedule(dynamic)
        for (int i=0; i<num_iterations; i++) {
            // random find and evaluate solution
            f_solution = evaluate_solution(solution);
            if (f_solution < f_best_solution_tmp) {  
                f_best_solution_tmp = f_solution;  
               // copy solution values to the best_solution_tmp array
            }
        }
        f_best_solution = f_best_solution_tmp;  
        #pragma omp barrier
        if(f_best_solution == f_best_solution_tmp){
            // copy best_solution_tmp values to the best_solution array
        }
    }
\end{verbatim}
	\caption{PRS antras algoritmas}
	\label {PRS2}
\end{figure}

\newpage

Abu algoritmai buvo ištestuoti su 2, 4 ir 6 branduoliais ir rezultatai matomi \textbf{\ref{dia:PRSdiagrama}} diagramoje:

\begin{diagram}[ht]
	\begin{center}
		\begin{tikzpicture}[domain=0:3, scale=1]
			\begin{axis}[
					xlabel={Branduolių skaičius},
					ylabel={Algoritmo efektyvumas},
					legend style={at={(1.5,1)},
							anchor=north,legend cell align=left} ]
				%x=y
				\addplot[black, mark=none] table [x=x, y=y,] {
						x y
						1 1
						2 2
						6 6
					};
				% Teorinis pagreitejimas
				\addplot[blue, mark=triangle*,
					nodes near coords={\pgfmathprintnumber[fixed zerofill, precision=3]{\pgfplotspointmeta}},
					visualization depends on=\thisrow{label} \as \Label] table [x=x, y=y] {
						x y label
						1 1 1
						2 1.49177515510631  1.49177515510631
						4 1.97818668769039  1.49177515510631
						6 2.21940844246444  2.21940844246444
					};
				% Realus pagreitejimasold algorithm

				\addplot [red, mark=triangle*,
					nodes near coords={\pgfmathprintnumber[fixed zerofill, precision=3]{\pgfplotspointmeta}},
					visualization depends on=\thisrow{label} \as \Label] table [x=x, y=y] {
						x y label
						1 1 1
						2 1.47144861531196 1.47144861531196
						4 1.91794196594863 1.91794196594863
						6 2.11142684438 2.11142684438
					};
				% Realus pagreitejimasold algorithm

				\addplot [orange, mark=triangle*,
					nodes near coords={\pgfmathprintnumber[fixed zerofill, precision=3]{\pgfplotspointmeta}},
					visualization depends on=\thisrow{label} \as \Label] table [x=x, y=y] {
						x y label
						1 1 1
						2 1.45710151139149 1.45710151139149
						4 1.92298859700795 1.92298859700795
						6 2.01390336583811 2.01390336583811
					};
				\legend {x=y,Teorinis pagreitėjimas,Realus pagreitėjimas(algortimas 1), Realus pagreitejimas(algoritmas 2)}
			\end{axis}
		\end{tikzpicture}
	\end{center}
	\caption{PRS algoritmo lygiagretinimo diagrama}
	\label{dia:PRSdiagrama}
\end{diagram}

Pastebime, kad \textbf{\ref{PRS1}} ir \textbf{\ref{PRS2}} algoritmai su 2 ir 4 branduoliais turėjo ganėtinai panašų efektyvumą:
\begin{table}[ht]
	\begin{center}
		\begin{tabular}{|c|c|c|}
			\hline
			Branduolių skaicius & 1 algoritmas           & 2 algoritmas            \\
			\hline
			2                   & \num{1.47144861531196} & \num{1.45710151139149}  \\
			\hline
			4                   & \num{1.91794196594863} & \num{1.922298859700795} \\
			\hline
			6                   & \num{2.11142684438}    & \num{2.01390336583811}  \\
			\hline
		\end{tabular}
	\end{center}
	\caption{Algoritmų efektuvumo priklausomybė nuo branduolių skaičiaus}
\end{table}

Tačiau, kai branduolių skaičius pasiekia 6, \ref{PRS2}-sis algoritmas nusileidžia efektyvumu \ref{PRS1}-jam. \\ \textbf{\ref{PRS1}-jį} algoritmą naudosime tolimesniuose 1 užduoties eksperimentuose.
\newpage
\subsection{Atstumų matricos skaičiavimas}

\begin{figure}[ht]
	\begin{verbatim}
    #pragma omp parallel for schedule(dynamic)
    for (int i=0; i<num_points; i++) {
        ...
        for (int j=0; j<=i; j++) {
            distance_matrix[i][j] = Haversine_distance(...);
        }
    }
\end{verbatim}
	\caption{Atstumų matricos skaičiavimas}
	\label {matricos algoritmas}
\end{figure}

\textbf{\ref{matricos algoritmas}} algoritmas, eksperimentiniu būdu buvo išbandytos įvairios lygiagretinimo strategijos (\textbf{Dynamic},\textbf{Guided},\textbf{ Static}). Gauti rezultatai matomi \textbf{\ref{dia:diagramaStrategijos}} diagramoje:

\begin{diagram}[ht]
	\begin{center}
		\begin{tikzpicture}[domain=0:3, scale=1]
			\begin{axis}[
					xlabel={Branduolių skaičius},
					ylabel={Algoritmo efektyvumas},
					legend style={at={(1.5,1)},
							anchor=north,legend cell align=left} ]
				\addplot[black, mark=none] table [x=x, y=y,] {
						x y
						1 1
						1.4 1.4
					};
				% Teorinis pagreitejimas
				\addplot[blue, mark=triangle*,
					nodes near coords={\pgfmathprintnumber[fixed zerofill, precision=3]{\pgfplotspointmeta}},
					visualization depends on=\thisrow{label} \as \Label] table [x=x, y=y] {
						x y label
						1 1 1
						2 1.20522833024383 1.20522833024383
						4 1.34304363117962  1.34304363117962
						6 1.39626355564776 1.39626355564776
					};
				%dynamix scheduling
				\addplot[red, mark=triangle*,
					nodes near coords={\pgfmathprintnumber[fixed zerofill, precision=3]{\pgfplotspointmeta}},
					visualization depends on=\thisrow{label} \as \Label] table [x=x, y=y] {
						x y label
						1 1 1
						2 1.17528298304024 1.17528298304024
						4 1.31106750410733 1.31106750410733
						6 1.38428949533786 1.38428949533786
					};
				%guided scheduling
				\addplot[orange, mark=triangle*,
					nodes near coords={\pgfmathprintnumber[fixed zerofill, precision=3]{\pgfplotspointmeta}},
					visualization depends on=\thisrow{label} \as \Label] table [x=x, y=y] {
						x y label
						1 1 1
						2 1.14934249820694 1.14934249820694
						4 1.27276503291191 1.27276503291191
						6 1.28140219409186 1.28140219409186
					};
				% static scheduling
				\addplot[violet, mark=triangle*,
					nodes near coords={\pgfmathprintnumber[fixed zerofill, precision=3]{\pgfplotspointmeta}},
					visualization depends on=\thisrow{label} \as \Label] table [x=x, y=y] {
						x y label
						1 1 1
						2 1.06525542297062 1.06525542297062
						4 1.24443394120132 1.24443394120132
						6 1.23684155937314 1.23684155937314
					};
				\legend {x=y,Teorinis pagreitėjimas,Dynamic realus pagreitėjimas,Guided realus pagreitėjimas, Static realus pagreitėjimas}
			\end{axis}
		\end{tikzpicture}
	\end{center}
	\caption{Įvairių strategijų efektyvumo diagrama}
	\label{dia:diagramaStrategijos}
\end{diagram}

Pastebime, jog \textbf{Dynamic} yra kur kas efektyvesnis, lyginant su \textbf{Static} ir \textbf{Guided} lygiagretinimo strategijomis ir kur kas labiau priartėja prie teorinio pagreitėjimo.

\textbf{Dynamic} direktyvą naudosime tolimesniuose 1 užduoties eksperimentuose.

\newpage
\subsection{Atstumų matricos ir PRS dalių lygiagretinimo rezultatai}
Iš ankstesnių eksperimentų radome, jog \textbf{\ref{PRS1}} PRS algoritmas ir \textbf{Dynamic} direktyva atstumų matricai skaičiuoti buvo efektyviausi sprendimo būdai.
Eksperimentiniu būdu ištestavus progamą, buvo gauti tokie rezultatai:
\begin{diagram}[ht]
	\begin{center}
		\begin{tikzpicture}[domain=0:3, scale=1]
			\begin{axis}[
					xlabel={Branduolių skaičius},
					ylabel={Algoritmo efektyvumas},
					legend style={at={(0.682,0.24)},
							anchor=north,legend cell align=left} ]
				%x=y
				\addplot[black, mark=none] table [x=x, y=y,] {
						x y
						1 1
						2 2
						4 4
						6 6
					};
				%teorinis pagreitėjimas
				\addplot[blue, mark=triangle*,
					nodes near coords={\pgfmathprintnumber[fixed zerofill, precision=3]{\pgfplotspointmeta}},
					visualization depends on=\thisrow{label} \as \Label] table [x=x, y=y] {
						x y label
						1 1 1
						2 1.99975760295228 1.99975760295228
						4 3.99854597016622 3.99854597016622
						6 5.99636580611994  5.99636580611994
					};
				% realus pagreitėjimas
				\addplot[red, mark=triangle*,
					nodes near coords={\pgfmathprintnumber[fixed zerofill, precision=3]{\pgfplotspointmeta}},
					visualization depends on=\thisrow{label} \as \Label] table [x=x, y=y] {
						x y label
						1 1 1
						2 1.91173481098984 1.91173481098984
						4 3.70599235210796 3.70599235210796
						6 4.44680617366648 4.44680617366648
					};
				\legend {x=y,Teorinis pagreitėjimas,Realus pagreitėjimas}
			\end{axis}
		\end{tikzpicture}
	\end{center}
	\caption{Efektyvumo diagrama, kai PRS ir matricos skaičiavimas lygiagretinami}
	\label{dia:diagramaLygiagretinimas}
\end{diagram}
\newpage
\section {2 laboratorinis darbas}
\subsection{Aprašymas}
\subsubsection{Užduotis}
Sudarytą bendros atminties lygiagretųjį algoritmą perdarykite į paskirstytos atminties algoritmą ir ištirkite jo pagreitėjimą naudodami 2, 4 ir 8 procesorius (naudokite uosis.mif.vu.lt skaičiavimų resursus). Ataskaitoje aprašykite kokiu būdu paskirstėte darbą skaičiavimo mazgams, kokiu būdu vyko komunikacija tarp jų.

Viename grafike pavaizduokite tiesinį pagreitėjimą ir eksperimentų būdu nustatytą atstumų matricos skaičiavimo pagreitėjimo, sprendinio skaičiavimo pagreitėjimo, bei viso algoritmo (kai matricos skaičiavimas ir sprendinio paieška išlygiagretinti) pagreitėjimo priklausomybes nuo procesorių skaičiaus.
\newpage
\subsection{PRS}
\subsubsection{Aprašas}
Kiekviena gija apskaičiuoja geriausią rezultatą kiekvienoje gijoje, tada 0-gija surenka visų giju rezultatus.
\subsubsection{Pseudokodas}
\begin{figure}[hbt!]
	\begin{verbatim}
    double f_solution, f_best_solution = 1e10;
    int it = 1;
    int iterations = num_iterations/ world_size;
    int remainder = num_iterations % world_size;
    iterations += remainder > world_rank ? 1 : 0;
    for (int i=0; i<iterations; i++) {
        // random find and evaluate solution
        f_solution = evaluate_solution(solution);
        if (f_solution < f_best_solution) { 
            f_best_solution = f_solution;
            if(f_best_solution == f_solution){
               // copy solution values to the best_solution array
            }
        }
    }
    if(world_rank == 0){
        // collect f_best_solution value for each thread, save best value and thread
        int rankId = -1;
        for(int j=1;j<world_size;j++){
            double response;
            MPI_Recv(&response, 1, MPI_DOUBLE, j, GET_BEST_SOLUTION_VALUE, MPI_COMM_WORLD, &stat);
            if(response < f_best_solution) {
                rankId = j;
                f_best_solution = response;
            }
        }

        if(rankId != 0 && rankId != -1){
            // retrieve best solution from thread
        }
        for(int j=1;j<world_size;j++){
            // stop worker threads
        }
    }
    else {
        do{
            MPI_Recv(&buff, 1, MPI_INT, 0, MPI_ANY_TAG, MPI_COMM_WORLD, &stat);
            if(stat.MPI_TAG!=STOP_THREADS){
                if(stat.MPI_TAG == GET_BEST_SOLUTION_VALUE) {
                    // returns f_best_solution
                }
                if(stat.MPI_TAG == GET_BEST_SOLUTION){
                    // returns best_solution
                }
            }
        } while(stat.MPI_TAG!=STOP_THREADS);
    }
\end{verbatim}
	\caption{PRS mpi pseudokodas}
	\label {PRS2}
\end{figure}

\newpage
\subsection{Atstumų matricos skaičiavimas}
\subsubsection{Aprašas}
Algoritmo principas toks:
\begin{enumerate}
\item gija- vadovas skirsto matricą darbininkams, vėliau surenka apskaičiuotus rezultatus, praneša, kad darbininkai gali baigti darbą(dėl to efektyvumas = gijų skaičius - 1). 
\item darbininkai gauna matricos eilutę, apskaičiuoja resultatą, gražina vadovui ir laukia, kada galės baigti savo darbą.
\end{enumerate}
\subsubsection{Pseudokodas}
\begin{figure}[hbt!]
	\begin{verbatim}
    distance_matrix = new double*[num_points];
    if(world_rank == 0){
        for (int i=0; i<num_points; i=i+(world_size-1)) {
            for(int j=1;j<world_size;j++){ // send to all threads array item to calculate
                if(i+j <=num_points){
                    int temp = i + j - 1;
                    MPI_Send(&temp, 1, MPI_INT, j, 0, MPI_COMM_WORLD);
                }
            }

            for(int j=1;j<world_size;j++){ 
                // receive calculated distance_matrix
            }
        }
        for(int j=1;j<world_size;j++){// stop slave thread work
            MPI_Send(&buff, 1, MPI_INT, j, STOP_THREADS, MPI_COMM_WORLD);
        }
    }
    else{
        do{
            MPI_Recv(&buff, 1, MPI_INT, 0, MPI_ANY_TAG, MPI_COMM_WORLD, &stat); 
            //receive matrix, calculate,ant return
            if(stat.MPI_TAG!=STOP_THREADS){
                double *localDistances = new double[buff + 1];
                localDistances ... = Haversine_distance(...)
                MPI_Send(localDistances, buff + 1, MPI_DOUBLE, 0, stat.MPI_TAG, MPI_COMM_WORLD);
            }
        } while(stat.MPI_TAG!=STOP_THREADS);

    }

    MPI_Barrier(MPI_COMM_WORLD);

    //Broadcast to all threads
\end{verbatim}
	\caption{Atstumų matricos mpi pseudokodas}
	\label {atstumų matricos pseudokodas}
\end{figure}

\newpage
\subsection{Atstumų matricos ir PRS dalių lygiagretinimos rezultatai}
Išlygiagretinus abi dalis gauname tokį grafiką:
\begin{diagram}[ht]
	\begin{center}
		\begin{tikzpicture}[domain=0:3, scale=1]
			\begin{axis}[
					xlabel={Branduolių skaičius},
					ylabel={Algoritmo efektyvumas},
					legend style={at={(0.5,-0.2)},
							anchor=north,legend cell align=left} ]
				%x=y
				\addplot[black, mark=none] table [x=x, y=y,] {
						x y
						1 1
						2 2
						4 4
						8 8
					};
				%Matricos  pagreitėjimas
				\addplot[blue, mark=triangle*,
					nodes near coords={\pgfmathprintnumber[fixed zerofill, precision=3]{\pgfplotspointmeta}},
					visualization depends on=\thisrow{label} \as \Label] table [x=x, y=y] {
						x y label
						1 1 1
						2 1.01775456330592 1.01775456330592
						4 2.25952942758542 2.25952942758542
						8 4.22252430634291 4.22252430634291
					};
				% PRS pagreitėjimas
				\addplot[red, mark=triangle*,
					nodes near coords={\pgfmathprintnumber[fixed zerofill, precision=3]{\pgfplotspointmeta}},
					visualization depends on=\thisrow{label} \as \Label] table [x=x, y=y] {
						x y label
						1 1 1
						2 1.88155803083376 1.88155803083376
						4 3.65909898281 3.65909898281
						8 7.3421580726395 7.3421580726395
					};
				% total
				\addplot[purple, mark=triangle*,
					nodes near coords={\pgfmathprintnumber[fixed zerofill, precision=3]{\pgfplotspointmeta}},
					visualization depends on=\thisrow{label} \as \Label] table [x=x, y=y] {
						x y label
						1 1 1
						2 	1.45183579993834 1.45183579993834
						4 3.00813425724229 3.00813425724229
						8 5.83292774333023 5.83292774333023
					};
				\legend {x=y,Matricos realus pagreitėjimas,PRS pagreitėjimas, Viso algoritmo pagreitėjimas}
			\end{axis}
		\end{tikzpicture}
	\end{center}
	\caption{Efektyvumo diagrama, kai PRS ir matricos skaičiavimas lygiagretinami}
	\label{dia:diagramaLygiagretinimas}
\end{diagram}
\newpage
\section {3 laboratorinis darbas}
\subsection{Aprašymas}
\subsubsection{Užduotis}
Sudarytą paskirstytos atminties lygiagretųjį algoritmą paleiskite (MIF superkompiuteryje) ir atlikite skaičiavimus naudodami 2, 4 ir 8 arba daugiau procesorių.

Viename grafike pavaizduokite tiesinį pagreitėjimą ir eksperimentų būdu nustatytą atstumų matricos skaičiavimo pagreitėjimo, sprendinio skaičiavimo pagreitėjimo, bei viso algoritmo (kai matricos skaičiavimas ir sprendinio paieška išlygiagretinti) pagreitėjimo priklausomybes nuo procesorių skaičiaus.
\newpage
\begin{diagram}[ht]
	\begin{center}
		\begin{tikzpicture}[domain=0:3, scale=1.5]
			\begin{axis}[
					xlabel={Branduolių skaičius},
					ylabel={Algoritmo efektyvumas},
					legend style={at={(0.5,-0.2)},
							anchor=north,legend cell align=left} ]
				%x=y
				\addplot[black, mark=none] table [x=x, y=y,] {
						x y
						1 1
						2 2
						4 4
						8 8
						16 16
						32 32
						64 64
					};
				%Matricos  pagreitėjimas
				\addplot[blue, mark=triangle*,
					nodes near coords={\pgfmathprintnumber[fixed zerofill, precision=3]{\pgfplotspointmeta}},
					visualization depends on=\thisrow{label} \as \Label] table [x=x, y=y] {
						x y label
						1 1 1
						2 0.913412979652526 0.913412979652526
						4 2.06791699512909 2.06791699512909
						8 2.97206997607846 2.97206997607846
						16 4.59570760063371 4.59570760063371
						32 4.87987625223182 4.87987625223182
						64 4.67179822092752 4.67179822092752
					};
				% PRS pagreitėjimas
				\addplot[red, mark=triangle*,
					nodes near coords={\pgfmathprintnumber[fixed zerofill, precision=3]{\pgfplotspointmeta}},
					visualization depends on=\thisrow{label} \as \Label] table [x=x, y=y] {
						x y label
						1 1 1
						2 1.95646741548835 1.95646741548835
						4 3.73598972103272 3.73598972103272
						8 6.45690863071238 6.45690863071238
						16 11.430633311221 11.430633311221
						32 16.324170626133 16.324170626133
						64 19.3207934293739 19.3207934293739
					};
				% total
				\addplot[purple, mark=triangle*,
					nodes near coords={\pgfmathprintnumber[fixed zerofill, precision=3]{\pgfplotspointmeta}},
					visualization depends on=\thisrow{label} \as \Label] table [x=x, y=y] {
						x y label
						1 1 1
						2 1.40296892943248 1.40296892943248
						4 2.92105063357079 2.92105063357079
						8 4.59271092791021 4.59271092791021
						16 7.54268369669204 7.54268369669204
						32 9.0037757691543 9.0037757691543
						64 9.25093080784449 9.25093080784449
					};
				\legend {x=y,Matricos realus pagreitėjimas,PRS pagreitėjimas, Viso algoritmo pagreitėjimas}
			\end{axis}
		\end{tikzpicture}
	\end{center}
	\caption{Efektyvumo diagrama, kai PRS ir matricos skaičiavimas lygiagretinami (leista ant MIF superkompiuterio)}
	\label{dia:diagramaLygiagretinimas}
\end{diagram}

\begin{diagram}[ht]
	\begin{center}
		\begin{tikzpicture}[domain=0:3, scale=1]
			\begin{axis}[
					xlabel={Branduolių skaičius},
					ylabel={Algoritmo efektyvumas},
					legend style={at={(0.5,-0.2)},
							anchor=north,legend cell align=left} ]
				%x=y
				\addplot[black, mark=none] table [x=x, y=y,] {
						x y
						1 1
						2 2
						4 4
						8 8
						16 16
					};
				%Matricos  pagreitėjimas
				\addplot[blue, mark=triangle*,
					nodes near coords={\pgfmathprintnumber[fixed zerofill, precision=3]{\pgfplotspointmeta}},
					visualization depends on=\thisrow{label} \as \Label] table [x=x, y=y] {
						x y label
						1 1 1
						2 0.913412979652526 0.913412979652526
						4 2.06791699512909 2.06791699512909
						8 2.97206997607846 2.97206997607846
						16 4.59570760063371 4.59570760063371
					};
				% PRS pagreitėjimas
				\addplot[red, mark=triangle*,
					nodes near coords={\pgfmathprintnumber[fixed zerofill, precision=3]{\pgfplotspointmeta}},
					visualization depends on=\thisrow{label} \as \Label] table [x=x, y=y] {
						x y label
						1 1 1
						2 1.95646741548835 1.95646741548835
						4 3.73598972103272 3.73598972103272
						8 6.45690863071238 6.45690863071238
						16 11.430633311221 11.430633311221
					};
				% total
				\addplot[purple, mark=triangle*,
					nodes near coords={\pgfmathprintnumber[fixed zerofill, precision=3]{\pgfplotspointmeta}},
					visualization depends on=\thisrow{label} \as \Label] table [x=x, y=y] {
						x y label
						1 1 1
						2 1.40296892943248 1.40296892943248
						4 2.92105063357079 2.92105063357079
						8 4.59271092791021 4.59271092791021
						16 7.54268369669204 7.54268369669204
					};
				\legend {x=y,Matricos realus pagreitėjimas,PRS pagreitėjimas, Viso algoritmo pagreitėjimas}
			\end{axis}
		\end{tikzpicture}
	\end{center}
	\caption{Efektyvumo diagrama, kai PRS ir matricos skaičiavimas lygiagretinami iki 16 branduolių (leista ant MIF superkompiuterio)}
	\label{dia:diagramaLygiagretinimas}
\end{diagram}




\end{document}
